\section{Judgment}
\label{sec:judgment}

The referees are the only persons permitted on the carpeted area (\ie the field and the border area).

\subsection{Head Referee}
\label{sec:head_referee}

The head referee is in charge of the game. Any decision of the head referee is valid. The head referee's decision is final and can not be changed afterwards, even by video proof. There is no discussion about decisions during the game, neither between the assistant referees and the head referee, nor between the audience or the teams and the head referee.

The head referee announces decisions by a clear loud call, and (as required) whistle sound.
The whistle, or where there is no whistle the first verbal word of the referees calls, defines the point in time at which the decision is made.
The referees should make efforts to use consistent and clear calls, and it is preferable for referees to use the calls as specified in these rules\footnote{The calls specified in these rules are detailed in English. With the agreement of the teams, the referees may use suitable calls in any language. The exception to this are technical challenge(s) that depends on the calls as specified. The use of consistent calls is also in preparation for future changes~(\cf Section~\ref{sec:future_changes})}.
The intention of specifying the referee calls is for clarity and consistency across games.

Where a whistle is required, the head referee first whistles and then announces the reason for the whistle.
The head referee may choose to use any normal sports whistle.
Each whistle sound should be short and not too loud as to interfere with other fields and simultaneous games.
The head referee must \textit{only} sound the whistle in circumstances described in these rules.
There are three circumstances when the whistle is sounded, Kick-off~(\cf Section~\ref{sec:kick-off}), a goal~(\cf Section~\ref{sec:goal}), and ending a half of gameplay~(\cf Section~\ref{sec:game_struct}).

The head referee should avoid handling the ball (except for placing a ball for kick-off), and avoid handling the robots.
Their duty is to monitor and adjudicate the game.
The head referee should only handle robots and the ball if absolutely necessary to expedite gameplay or removal of penalised robots, where the assistant referees are otherwise occupied or too far away.

\subsection{Assistant Referees}
\label{sec:assist_referee}
The two assistant referees handle the robots and the ball. They start the robots if the wireless is not working, they move the robots, if manual placement is requested, they take the robots out when they are penalized, and they put the robots in again. If a team requests to pick up a robot, an assistant referee will pick it up and give it to one of the team members once the head referee approves. An assistant referee will also put the robot back on the field. An assistant referee will also replace the ball when it goes off the field or becomes stuck between a players feet.

The assistant referees can \textit{indicate} violations against the rules committed by robots to the head referee, so that the head referee can decide whether to penalize a certain robot or not. Assistant referees should only enter the field to execute a decision made by the main referee. They should not prevent robots from falling during the game.

\subsection{Operator of the GameController}
\label{sec:gameControllerOp}
The operator of the GameController sits at a PC outside the playing area.
As with the head referee, the operator should make efforts to use consistent and clear calls.
They will signal any change in the game state to the robots via the wireless as they are announced by the head referee.
Note that for both kick-offs and goals, the moment of whistling is determining, not the verbal announcement of the head referee.
The operator will also inform the assistant referees when a timed penalty is over and a robot has to be placed back on the field.
They should announce when the ball is in play on kick-off by stating ``Ball Free'', if the \KickOffBallFreeTime time period has elapsed in the playing state.
They are also responsible for keeping the time of each half (\ie, they stop the clock after a goal or game stuck, and continues it at the kick-off\footnote{The clock may not be stopped during the preliminaries.}).
They should count aloud the remaining seconds in a half once the time remaining is 5 seconds or less.
Finally, they should repeat the calls of the head referee to make sure it was heard correctly.

\subsection{Referees During the Match}

The head referee and the assistant referees should wear clothing and socks \emph{of black or dark blue color} (blue jeans are acceptable) and avoid reserved colors for the ball, the goals, and player markings in their clothing. They may enter the field in particular situations, \eg, to remove a robot when applying a penalty. They should avoid interfering with the robots as much as possible.

\subsection{A Remark on Artificial Landmarks}
\label{sec:judgment:landmarks}

The head referee may decide at any point before or during a game to relocate any objects around the field, or direct persons to another position around the field.

The intent of using same-colored goals is to remove artificial landmarks.
Robots should be able to localize with the SPL field and its ``normal'' surroundings.
Introducing new team-specific artificial landmarks is against the spirit and intention of the league's progress.
The application of this rule needs to be well considered and should be reserved for situations which seem constructed by one team or another, but will ultimately be the head referee's decision alone.
