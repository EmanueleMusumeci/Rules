\documentclass{article}
\usepackage{graphicx}
\usepackage{url}
%\usepackage{hyperref}
\usepackage{verbatim}
%\usepackage{times,mathptm}
\usepackage{times}
% the following package is optional:
%\usepackage{latexsym}
%\usepackage{picins}
\usepackage{url}
\usepackage{epsfig}
\usepackage{amsmath}
%\usepackage{macros}
%\usepackage{small-caption}
%\usepackage[l]{floatflt}

\begin{document}

\title{Technical Challenges for the RoboCup 2011 Standard Platform League Competition}

\author{RoboCup SPL Technical Committee}

\maketitle

\section{Introduction}
\label{sec:introduction}

The RoboCup 2011 Standard Platform League Competition is going to have only 
one technical challenge, which is the Open Challenge.

The team with the top score in the challenge will receive 28 points, each 
position thereafter will receive 1 less point; i.e. 1st = 28pts, 2nd = 27pts, 
3rd = 26pts ... 28th = 1pt. In the case of a draw, each team will receive 
the average of the points allocated to these positions; e.g. if three team 
tie for 2nd, they will receive $(27+26+25)/3 = 26$ points. Teams not competing 
in the challenge will receive 0 points, also if a team competes but fails to 
score a point (or receive a vote) they will receive 0 points again. The team 
with the highest score after the challenge is deemed the challenge winner.

The challenge will use the 2011 field and the 2011 rules will apply.


\section{The Open Challenge}
\label{sec:open}
\newcommand{\openMinNum}{three}

This challenge is designed to encourage creativity within the Standard 
Platform League, allowing teams to demonstrate interesting research in 
the field of autonomous systems. Each team will be given \openMinNum{} 
minutes of time on the RoboCup field to demonstrate their research. 
Each team \emph{should} also distribute a short, one page description of 
their research prior to the competitions. The winner will be decided by 
a vote among the entrants. In particular:

\begin{itemize}
\item 
Teams must describe the content of their demonstration to the technical 
committee at least \emph{four weeks} before the competitions. 
\item 
The demonstration should be strongly related to the scope of the league. 
Irrelevant demonstrations, such as dancing and debugging tool presentations 
are discouraged.
\item 
Each team may use any number of Aldebaran Nao robots. Teams must arrange
for their own robots.
\item 
Teams have \openMinNum{} minutes to demonstrate their research. This
includes any time used for initial setup. Any demonstration deemed
likely to require excessive time may be disallowed by the technical
committee.
\item 
Teams may use extra objects on the field, as part of their
demonstration. \emph{Robots other than the Naos may not be used}.
\item 
The demonstration must \emph{not} mark or damage the field. Any
demonstration deemed likely to mark or damage the field may be
disallowed by the technical committee.
\item 
The demonstration may \emph{not} use any off-board sensors or
actuators, or modify the Nao robots.
\item 
The demonstration may use off-board computing power connected over the
wireless LAN. This is the only challenge in which off-board
computation is allowed.
\item 
The demonstration may use off-board human-computer interfaces. This
is the only challenge in which off-board interfaces, apart from the
Game Controller, are allowed.
\end{itemize}

The winner will be decided by a vote among the entrants using a Borda
count (\url{http://en.wikipedia.org/wiki/Borda_count}). Each participating 
team will list their top 10 teams in order (excluding themselves).
The teams are encouraged to evaluate the performance based on the
following criteria: technical strength, novelty, expected impact and
relevance to RoboCup. At a time decided by the designated referee,
within 30 minutes of the last demonstration if not otherwise
specified, the captain of each team will provide the designated
referee with their rankings. Each ranking is converted to points based on 
the scoring criteria mentioned in Section \ref{sec:introduction}. Any 
points awarded by a team to itself will be disregarded. The points awarded 
by the teams are summed and the team with the highest total score shall be 
the winner.


\end{document}


