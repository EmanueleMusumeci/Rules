\documentclass[12pt]{article}

\usepackage{times,fullpage,xspace,fancyhdr,url}
\usepackage[pdftex]{graphicx}
\usepackage[pdftex,
            a4paper,
            colorlinks=true,
            urlcolor=black,
            linkcolor=black,
            citecolor=black,
            bookmarksopen=false,
            bookmarksnumbered=true,
            pdfstartview=FitH]{hyperref}

\usepackage{graphicx}
\usepackage{xspace,color}
\pdfcompresslevel=9
\newcommand{\leaguename}{RoboCup Standard Platform League (NAO) }
\hypersetup{
 pdftitle={\leaguename Technical Challenges 2016},
 pdfauthor={Technical Committee},
}
\usepackage[latin1]{inputenc}
\usepackage{amsmath}
\usepackage{times}

% comment 'disable' in to disable all the todo notes :)
\usepackage
[
%disable
]{todonotes}

\sloppy
\newcommand{\ie}{\mbox{i.\,e.}\xspace}
\newcommand{\eg}{\mbox{e.\,g.}\xspace}
\newcommand{\cf}{\mbox{cf.}\xspace}
\newcommand{\comment}[1]{\marginpar{\pdfannot width 4in height .5in depth 8pt {/Subtype /Text /Contents (#1)}}}
\newcommand{\inparagraph}[1]{\paragraph{#1\hspace{-1em} }}

\long\def\commentk#1{{\bf ++K: #1++}}

% some colors
\definecolor{orange}{rgb}{1,0.5,0}
\definecolor{red}{rgb}{1,0,0}
\definecolor{green}{rgb}{0,1,0}


\title{\leaguename \\ Technical Challenges}
\author{RoboCup Technical Committee}
\date{(2016 technical challenge rules, as of \today)}

\setlength{\parindent}{0pt}
\setlength{\parskip}{6pt plus 6pt minus 3 pt}
\setcounter{tocdepth}{1}
\widowpenalty=10000
\clubpenalty=10000

\pagestyle{fancy}
\lhead{}
\chead{}
\rhead{}
\lfoot{}
\cfoot{}
\rfoot{}

\renewcommand{\headrulewidth}{0.4pt}
\renewcommand{\footrulewidth}{0.4pt}

% needed to align an image and text correctly side by side
\newcommand{\imagebox}[1]{\raisebox{2ex}{\raisebox{-\height}{#1}}}



\begin{document}

\maketitle

At RoboCup 2016, the Standard Platform League will hold two different technical challenges, which are described in this document.

The scores earned in each challenge will vary in magnitude.  Hence, they must be scaled before calculating the overall technical challenge rankings.  Teams who do not participate in a challenge will receive 0 points for that challenge.  The team with the highest total score for a challenge will get 25 points for that challenge, while the team with the lowest total score for a challenge will get 5 points for that challenge.  A linear equation will then be fit to these two points, and each other participating team in that challenge will gain points for that challenge based on this equation.

For both challenges, no changes of code or configuration are allowed for any participating team after the first team starts the challenge. 

Questions or comments on these rules should be mailed to {\small \url{rc-spl-tc@lists.robocup.org}}.

\vfill

\renewcommand\contentsname{Challenges}
\tableofcontents
\setcounter{tocdepth}{1}

\thispagestyle{fancy}

\clearpage

\cfoot{\thepage}
\setcounter{page}{1}

\newcommand{\openMinNum}{three}


% % % % % % % % % % % % % % % % % % % % % % % %


\section{No Wi-Fi Challenge}

The purpose of this challenge is to foster communication without the use of wireless networks.
As in human soccer, communications may be visual, acoustic, or some combination of both.

\subsection{Setup}
Each participating team has to provide two robots. Insofar as it may affect visual communications,
the robots may wear jerseys if so desired and the
jersey colour and pattern may be chosen freely in this case. 

One of the robots shall be designated as
the communication transmitter and the other designated as the communication receiver. 
These robots shall be referred to as T and R respectively.
Both robots will be connected via long ethernet cables to the game controller computer.
The robots will not be able to communicate directly with each other by any means except the non-WiFi communications mechanism. In particular, the WLAN interface on the robots must be disabled for the challenge.
A communications tester tool running on the game controller computer will be used to send messages to T and receive forwarded messages from R. 
All communications between the game controller computer and T or R shall be via TCP/IP over ethernet.

Two message types will be used in the challenge: the location message and the data message. The location message is intended to be representative of some typical information that robots may need to share. The location message  payload consists of a field x,y position defined using integer values in millimeters relative to the SPL standard message coordinate frame.
The data message is used to evaluate communication of arbitrary binary data. The data message payload consists of an integer data length followed by the specified number of binary bytes. The maximum data length will be 10000 bytes. (The detailed message format C code for both message types will be provided at a later time on the SPL website.)

The robots will be placed by the technical committee. 
R shall be placed in the manually placed goalie
position, that is, in the center of the penalty box with its feet immediately in front of the end line (position 1).
T shall be placed facing R at one of the following positions: the field centre spot (position 2) or the far away penalty spot (position 3). It will initially be placed in position 3.

During setup, robots shall be in the penalized state under chest button control. No communication may take place while in this state.

\subsection{Procedure}
The game state is changed to playing on both robots using the chest button.

The communications tester sends a location message to T which should immediately transmit it to R.
If R receives the location message successfully it should forward it (via ethernet) to the communication tester. The robot must also point at the corresponding spot on the field for 3 seconds using its right or left arm. Once R has forwarded the message or pointed to the spot on the field, the message will considered received and another message may be sent. A message will be considered lost if, after a message being sent by the communications tester, (a) T does not initiate communication within 5 seconds, or (b) R does not point or forward the received message to the game controller computer within 15 seconds.

Three such messages will be sent with T in position 3. A further 3 messages will be sent with T in postion 2.

Not all communication schemes are suitable for general purpose data communications. The second part of the non-WiFi challenge investigates the suitability for general purpose data communications by estimating the practical data rate that can be achieved. This test will be performed with T in position 2.
In order to test the possible data rate using the non-WiFi communication scheme, the communications tester will send a single 10000 byte message to T. T should estimate the number of bytes, N, that it can transmit in 15 seconds using its non-WiFi communications scheme. Thereafter, T must transmit N or the number of bytes originally received (whichever is less) to R. The data may be fragmented as necessary for the non-WiFi channel. R should forward each fragment received immediately to the communications tester as a separate data message. If communication continues after 15 seconds, subsequent messages will be ignored by the communications tester. The estimated raw data rate will be the difference between the time at which the last data message was received by the communications tester and the time at which the large data message was originally sent divided by the total number of bytes received by the communications tester. The error rate (which should ideally be zero) will be calculated as the number of incorrectly received bytes divided by the total number of bytes received. (Received bytes are compared to transmitted bytes and any mismatches are considered to be incorrectly received bytes.)

Communication of the data message is considered to have failed if, after the data message being sent by the communications tester, (a) T does not initiate communication within 5 seconds, or (b) R does not forward any data to the communications tester within 15 seconds.

\subsection{Score}
Scores will be allocated separately for communication of the location messages and communication of the data message.

A location message is considered to be correctly received if R points at the correct spot or R forwards the a correct location message back to the communications tester. If R does both, then the correctness of the forwarded location message takes precedence.
Each correctly received location message from position 3 will be awarded three points. Each correctly received location message from position 2 will be awarded two points. No points are awarded for incorrectly received location messages.

Scoring of the data message communication is determined by the raw data rate and error rate. The maximum raw data rate achieved by any team is awarded 10 points. All other scores are linearly scaled by raw data rate as $rateScore = 10 \times \frac{rawDataRate}{maxDataRate}$. Up to 5 additional points may be awarded depending on the error rate. As the error rate ranges between 0 and 1, the error rate points are calculated as $errorScore = 5\times (1 - errorRate)$. 

\newpage
% % % % % % % % % % % % % % % % % % % % % % % %



\section{Outdoor Challenge}

\subsection{Purpose}

The purpose of the challenge is to foster skills for playing in natural lighting on a more realistic playing surface.

\subsection{Setup}
TODO

\subsection{Procedure}
TODO 

\subsection{Score}
TODO

\end{document}

