\documentclass[12pt]{article}

\usepackage{times,fullpage,xspace,fancyhdr,url}
\usepackage[pdftex]{graphicx}
\usepackage[pdftex,
            a4paper,
            colorlinks=true,
            urlcolor=black,
            linkcolor=black,
            citecolor=black,
            bookmarksopen=false,
            bookmarksnumbered=true,
            pdfstartview=FitH]{hyperref}

\usepackage{graphicx}
\usepackage{xspace,color}
\pdfcompresslevel=9
\newcommand{\leaguename}{RoboCup Standard Platform League (NAO) }
\hypersetup{
	pdftitle={\leaguename Technical Challenges},
	pdfauthor={Technical Committee},
}
\usepackage[latin1]{inputenc}
\usepackage{amsmath}

% comment 'disable' in to disable all the todo notes :)
\usepackage
[
%disable
]{todonotes}

\sloppy
\newcommand{\ie}{\mbox{i.\,e.}\xspace}
\newcommand{\eg}{\mbox{e.\,g.}\xspace}
\newcommand{\cf}{\mbox{cf.}\xspace}
\newcommand{\comment}[1]{\marginpar{\pdfannot width 4in height .5in depth 8pt {/Subtype /Text /Contents (#1)}}}
\newcommand{\inparagraph}[1]{\paragraph{#1\hspace{-1em} }}


% some colors
\definecolor{orange}{rgb}{1,0.5,0}
\definecolor{red}{rgb}{1,0,0}
\definecolor{green}{rgb}{0,1,0}

\title{\leaguename\\Technical Challenges}
\author{RoboCup Technical Committee}
\date{(2019 technical challenge rules draft, as of \today)}

\setlength{\parindent}{0pt}
\setlength{\parskip}{12pt plus 6pt minus 3 pt}
\setcounter{tocdepth}{1}
\widowpenalty=10000
\clubpenalty=10000

\pagestyle{fancy}
\lhead{}
\chead{}
\rhead{}
\lfoot{}
\cfoot{}
\rfoot{}

\renewcommand{\headrulewidth}{0.4pt}
\renewcommand{\footrulewidth}{0.4pt}

\newcommand{\TotalWidth}{7.4~m\xspace}
\newcommand{\TotalLength}{10.4~m\xspace }
\newcommand{\KickOffAutoTime}{45 seconds\xspace}
\newcommand{\FreeKickTime}{30 seconds\xspace}
\newcommand{\FreeKickRadius}{0.75m\xspace}
\newcommand{\PenaltyKickTime}{30 seconds\xspace}

% needed to align an image and text correctly side by side
\newcommand{\imagebox}[1]{\raisebox{2ex}{\raisebox{-\height}{#1}}}

\begin{document}

\maketitle

At RoboCup 2019, the Standard Platform League will hold two different technical challenges, which are described in this document. Note that this is a preliminary draft of the technical challenges, and hence small details and scoring may be altered. However, any alterations will be made and announced well before the 2019 competition.

The scores earned in each challenge will vary in magnitude. Hence, they must be scaled before calculating the overall technical challenge rankings. Teams who do not participate in a challenge will receive 0 points for that challenge. The team with the highest total score for a challenge will get 25 points for that challenge, while the team with the lowest total score for a challenge will get 5 points for that challenge. A linear equation will then be fit to these two points, and each other participating team in that challenge will gain points for that challenge based on this equation.

For both challenges, no changes of code or configuration are allowed for any participating team after the first team starts the challenge.

Questions or comments on the technical challenge rules should be mailed to \url{rc-spl-tc@lists.robocup.org}.

\vfill
\tableofcontents
\setcounter{tocdepth}{3}
\thispagestyle{fancy}
\clearpage
\cfoot{\thepage}
\setcounter{page}{1}

\section{Open Research Challenge}

The open research challenge serves as a supplement for the posters that are mentioned in Appendix A.1 of the official rules document. It should be especially used to demonstrate a team's own contributions to the state of the art in the league. Participation in this challenge is mandatory. In its procedure, it is similar to the Open Challenges that have been held until RoboCup 2014.

Each team \emph{must} send a short, maximum one page document describing what kind of demonstration they do during their presentation time to the technical committee by \textbf{June 1, 2019}. This document will be used to judge whether the presentation is adherent to the rules listed below. The technical committee will review the documents and notify teams whether their proposed demonstrations are acceptable. Teams who do not submit this description by the deadline will only be allowed to hold a presentation.

Each team will be given three minutes of time on an SPL field for a presentation. Teams may also use the large mixed team field for their demonstration (but should indicate it in their description to avoid switching fields more than once).

\begin{itemize}
\item The presentation must be strongly related to the scope of the league. Irrelevant demonstrations, such as dancing and debugging tool presentations, are not allowed.
\item The presentation should involve a demonstration using (any number of) real NAO robots (\ie presentations without practical demonstration are assumed to be ranked lower than others). The use of simulation is also possible, but only if the demonstrated research is not related to topics inherently interacting with the real world (\eg walking, perception) and it is guaranteed, \eg through interactivity, that it is actually a simulation and not a video, and the simulated robot resembles a NAO robot (\ie it does not have obvious capabilities that the NAO does not have).
\item Teams have three minutes for their presentation. At most one additional minute may be used for initial setup. Any demonstration deemed likely to require excessive time will be disallowed by the technical committee.
\item Teams may use extra objects on the field as part of their demonstration. \emph{Robots other than the NAOs may not be used.}
\item A demonstration must \emph{not} mark or damage the field. Any demonstration deemed likely to mark or damage the field will be disallowed by the technical committee.
\item A demonstration may \emph{not} modify the NAO robots.
\end{itemize}

\subsection{Score}

The winner will be decided by a vote among the SPL teams using the Borda count mechanism (\url{http://en.wikipedia.org/wiki/Borda_count}). Each SPL team will vote for their top 10 teams in order (excluding themselves). Teams are encouraged to evaluate the performance based on the following criteria: technical strength, novelty, expected impact and relevance to RoboCup, quality of the presentation. At a time decided by the designated referee, within one hour of the last demonstration if not otherwise specified, the captain of each team will submit his or her team's rankings by filling out an online form. Any points awarded by a team to itself will be disregarded. The points awarded by the teams will be summed and thus form the score of this challenge which is then converted according to the formula described in the preface of this document.

\newpage

\section{Directional Whistle Challenge}

The intention of this challenge is to investigate the possibilities of localizing the point where the referee whistle is blown.

\subsection{Setup}

Each participating team provides a number of robots from \(1\) to \(5\) for this challenge. The choice of the actual number of robots is up to the team, but it might affect the score that is awarded. The robots are placed on predefined spots on a regular SPL field and oriented in some predefined orientation (per spot). The exact poses will be announced well before the challenge, such that teams can make sure that their robots know where they are during the challenge.

To evaluate the success of the challenge, one of the robots must be in a WiFi network (details will be made available at the competition) connected via TCP/IP to a testing application (which will be released before the competition). Wireless communication between the robots using the provided network is also allowed.

The robots have to be in a manually penalized state when handed over to the TC. The GameController will not be used in this challenge. The robots will be manually unpenalized with a single chest button press (not necessarily at the same time for all robots) after they have been placed on the field by TC members. Only then should a robot connect to the testing application.

\subsection{Procedure}
The challenge is conducted for each team separately after each other. After the robots have been set up, a ``referee'' blows a whistle from \(8\) different points (that are the same for all teams). Some of them are on the same field as the robots and others are on neighboring fields.

In the moment the referee blows the whistle, the operator of the testing application presses a button so that incoming messages will be associated with the current whistle position and the starting time of the attempt can be tracked. Following the whistle, the testing application will accept the first message it receives until up to \(5\) seconds after the attempt has started. Messages that are received in between attempts will be ignored by the testing application. After \(5\) seconds, the attempt is over and the referee will proceed to the next whistle position.

The exact message format will be specified together with the release of the testing application, but can be expected to contain two fields:
\begin{itemize}
\item One boolean flag that indicates only whether the whistle is assumed to be one the same field or another field.
\item The 2D position of the whistle on the ground in Cartesian coordinates, relative to a coordinate system that has its origin in the center of the field.
\end{itemize}
The message will be used by the testing application to automatically calculate the score, as it has access to the list of positions occupied by robots and the list of points from which the whistle is blown.

Note that though the message contains Cartesian coordinates, the scoring uses polar coordinates to better model the deviation of measurements. For this, the pose of the robot on the field that is closest to the actual whistle is used as reference pose. The testing application will automatically do the necessary transformations.

The robots are only allowed to move their heads during the challenge. All other motions (especially locomotion) are not allowed and result in the disqualification of that team from this challenge. Just as in the normal rules, any sports whistle is allowable. It can be assumed that the whistle will be blown by a standing person.

\subsection{Score}
Each whistle attempt contributes to the score of a team based on three aspects:
\begin{description}
\item[Whether the decision between ``same field'' and ``other field'' is correct:] \(1\) point is awarded if it is correct, \(0\) otherwise.
\item[Angular precision:] \(1\) point is awarded if the transmitted angle is within a \(\pm15^\circ\) cone of the actual direction. \(0.5\) points are awarded if it is within the \(\pm15^\circ\) cone that points in the opposite direction from the reference pose or is within a \(\pm25^\circ\) cone of the actual direction.
\item[Distance precision:] \(1\) point is awarded if the transmitted distance is within a \(\pm10\%\) range of the actual distance. \(0.5\) points are awarded if it is within \(\pm20\%\) of the actual distance.
\end{description}
That is, if the whistle is correctly detected as being on the same field, the direction is also correct but the distance is wrong, \(2\) points are awarded. If, after \(5\) seconds, there is no reaction from the robot, the attempt is over, no points are awarded, and the whistle is blown from the next of the \(8\) points.

The scores of all \(8\) attempts are added together to form a number between \(0\) and \(24\) which is the total score of the team. In case of a tie (after transforming the score of this challenge to the overall challenge score and combining it with the Open Research Challenge), teams which used less robots in this challenge will be ranked superior.

\end{document}
