\usepackage{times,fullpage,xspace,fancyhdr,url}
\usepackage[pdftex]{graphicx}
\usepackage[pdftex,
            colorlinks=true,
            urlcolor=black,
            linkcolor=black,
            citecolor=black,
            bookmarksopen=false,
            bookmarksnumbered=true,
            pdfstartview=FitH]{hyperref}

\usepackage{graphicx}
\usepackage{xspace,color}
\pdfcompresslevel=9
\newcommand{\leaguename}{RoboCup Standard Platform League (NAO) }
\hypersetup{
 pdftitle={\leaguename Rule Book},
 pdfauthor={Technical Committee SPL},
}
\usepackage{microtype}
\usepackage[utf8]{inputenc}
\usepackage{amsmath}
\usepackage{xargs}
\usepackage{enumitem}
\usepackage[colorinlistoftodos,prependcaption,textsize=tiny]{todonotes}
\newcommandx{\unsure}[2][1=]{\todo[linecolor=red,backgroundcolor=red!25,bordercolor=red,#1]{#2}}
\newcommandx{\change}[2][1=]{\todo[linecolor=blue,backgroundcolor=blue!25,bordercolor=blue,#1]{#2}}
\newcommandx{\info}[2][1=]{\todo[linecolor=green,backgroundcolor=green!25,bordercolor=green,#1]{#2}}
\newcommandx{\improvement}[2][1=]{\todo[linecolor=yellow,backgroundcolor=yellow!25,bordercolor=yellow,#1]{#2}}

% comment 'disable' in to disable all the todo notes :)
\usepackage
[
%disable
]{todonotes}

\sloppy
\newcommand{\ie}{\mbox{i.\,e.}\xspace}
\newcommand{\eg}{\mbox{e.\,g.}\xspace}
%\newcommand{\cf}{\mbox{cf.}\xspace}
\newcommand{\cf}{see\xspace}
\newcommand{\comment}[1]{\marginpar{\pdfannot width 4in height .5in depth 8pt {/Subtype /Text /Contents (#1)}}}
\newcommand{\inparagraph}[1]{\paragraph{#1\hspace{-1em} }}

\setlength{\parindent}{0pt}
\setlength{\parskip}{12pt plus 6pt minus 3 pt}
\setcounter{tocdepth}{1}
\widowpenalty=10000
\clubpenalty=10000

\pagestyle{fancy}
\lhead{}
\chead{}
\rhead{}
\lfoot{}
\cfoot{}
\rfoot{}

% some colors
\definecolor{orange}{rgb}{1,0.5,0}
\definecolor{red}{rgb}{1,0,0}
\definecolor{green}{rgb}{0,1,0}


% needed to align an image and text correctly side by side
\newcommand{\imagebox}[1]{\raisebox{2ex}{\raisebox{-\height}{#1}}}

% !TeX root = ../SPL-Rules.tex
% !TeX spellcheck = en_US
\newcommand{\TotalWidth}{7.4}
\newcommand{\TotalLength}{10.4}
\newcommand{\GoalScoredDelay}{15}
\newcommand{\KickOffAutoTime}{45}
\newcommand{\KickOffBallFreeTime}{10}
\newcommand{\FreeKickTime}{30}
\newcommand{\FreeKickRadius}{0.75}
\newcommand{\PlayingDelayTime}{15}
\newcommand{\PenaltyFreeKickTime}{30}
\newcommand{\PenaltyFreeKickSetupTime}{30}
\newcommand{\PenaltyKickTime}{30}
\newcommand{\StandardPenaltyTime}{45}
\newcommand{\StandardPenaltyIncrease}{10}
\newcommand{\NovelContributionTime}{3 years\xspace}
\newcommand{\GameStuckTime}{30}
\newcommand{\TeamMessageLimit}{1200} % Limit of number of packets available to one team during a game with two halves of 10 minutes.
\newcommand{\TeamMessageLimitMinute}{60} % Limit for the average number of packets available to one team during a minute of gameplay.



\newlength\colorboxwidth
\setlength{\colorboxwidth}{\dimexpr\textwidth-2\fboxsep}
\newlength\colorboxwidthdefinition
\setlength{\colorboxwidthdefinition}{\dimexpr\textwidth-11\fboxsep}
\newlength\colorboxwidthfigure
\setlength{\colorboxwidthfigure}{\dimexpr\textwidth-17\fboxsep}
\newcommand{\cbp}[1]{\colorbox{blue!25}{\parbox{\colorboxwidth}{#1}}}
\newcommand{\cbd}[1]{\colorbox{blue!25}{\parbox{\colorboxwidthdefinition}{#1}}}
\newcommand{\cbf}[1]{\colorbox{blue!25}{\parbox{\colorboxwidthfigure}{#1}}}
\newcommand{\cbw}[1]{\colorbox{blue!25}{#1}}

\usepackage{listings}
\usepackage{xcolor}

\definecolor{eclipseStrings}{RGB}{42,0.0,255}
\definecolor{eclipseKeywords}{RGB}{127,0,85}
\colorlet{numb}{magenta!60!black}
\definecolor{delim}{RGB}{20,105,176}
\colorlet{punct}{red!60!black}

\lstdefinelanguage{json}{
    basicstyle=\normalfont\ttfamily,
    commentstyle=\color{eclipseStrings}, % style of comment
    stringstyle=\color{eclipseKeywords}, % style of strings
    numbers=left,
    numberstyle=\scriptsize,
    stepnumber=1,
    numbersep=8pt,
    showstringspaces=false,
    breaklines=true,
    frame=lines,
    % backgroundcolor=\color{gray}, %only if you like
    string=[s]{"}{"},
    comment=[l]{:\ "},
    morecomment=[l]{:"},
    literate=
        *{0}{{{\color{numb}0}}}{1}
         {1}{{{\color{numb}1}}}{1}
         {2}{{{\color{numb}2}}}{1}
         {3}{{{\color{numb}3}}}{1}
         {4}{{{\color{numb}4}}}{1}
         {5}{{{\color{numb}5}}}{1}
         {6}{{{\color{numb}6}}}{1}
         {7}{{{\color{numb}7}}}{1}
         {8}{{{\color{numb}8}}}{1}
         {9}{{{\color{numb}9}}}{1}
         {:}{{{\color{punct}{:}}}}{1}
        {,}{{{\color{punct}{,}}}}{1}
        {\{}{{{\color{delim}{\{}}}}{1}
        {\}}{{{\color{delim}{\}}}}}{1}
        {[}{{{\color{delim}{[}}}}{1}
        {]}{{{\color{delim}{]}}}}{1}
}

\usepackage[edges]{forest}
\definecolor{foldercolor}{RGB}{124,166,198}

\tikzset{pics/folder/.style={code={%
    \node[inner sep=0pt, minimum size=#1](-foldericon){};
    \node[folder style, inner sep=0pt, minimum width=0.3*#1, minimum height=0.5*#1, above right, xshift=0.05*#1] at (-foldericon.west){};
    \node[folder style, inner sep=0pt, minimum size=#1] at (-foldericon.center){};}
    },
    pics/folder/.default={16pt},
    folder style/.style={draw=foldercolor!80!black,top color=foldercolor!40,bottom color=foldercolor}
}

\forestset{is file/.style={edge path'/.expanded={%
        ([xshift=\forestregister{folder indent}]!u.parent anchor) |- (.child anchor)},
        inner sep=1pt},
    this folder size/.style={edge path'/.expanded={%
        ([xshift=\forestregister{folder indent}]!u.parent anchor) |- (.child anchor) pic[solid]{folder=#1}}, inner xsep=0.6*#1},
    folder tree indent/.style={before computing xy={l=#1}},
    folder icons/.style={folder, this folder size=#1, folder tree indent=3*#1},
    folder icons/.default={10pt},
}