\section{Challenge 3: 1 vs 1}

\subsection{Idea of Challenge}
% Patrick
The aim of this challenge is to have some competitive game play given the current situation.   

This challenge focuses on the following aspects:
\begin{itemize}
    \item Remote or autonomous deployment of NAO software and standardised settings for game play in a remote arena on foreign robots
    \item Automatic and semi-automatic calibration of NAO (vision, motion, etc.) 
    \item One versus One NAO competition in a ladder KO competition with all teams without much robot interaction. 
\end{itemize}

Requirements to participate in the scored part of this challenge are being able to
\begin{itemize}
    \item host an arena
    \item deploy your robot's software remotely or fully autonomous setup. 
\end{itemize}

If teams provide an autonomous setup all teams can use this to play against this in their own arena. Also teams who cannot host an arena.

\subsection{Prerequisites}
% Patrick
For this challenge teams need to fulfil multiple requirements, which are listed below in more detail.

First to be able to participate Teams have to  be able to deploy their software on the NAO remotely or you can deliver an autonomous setup. There will be several opportunities for the teams to talk, discuss, exchange ideas and code over the next few month until RoboCup. Two options will be properly available: Meetings like RoDEO and an spring RoHOW as well as to discuss on the SPL Discord channel. The second option is a code sharing section as it is available for the V6 support on RC SPL website.  

If teams are not able to participate, they (and the other teams as well) can download and deploy the images for fully autonomous setup, calibration and challenge play in the own lab. These games are not part of the ladder system and are outside of the competition.

\subparagraph*{Basic requirements for teams}

\begin{itemize}
    \item A team must be able deploy their robot software remotely or be able to produce a fully autonomous image for a robot
    \item A team must be able to host an arena. If not they have to find an substitute team, who takes over their hosting arena responsibilities.
    \item A robot must be able to semi- or fully automatic calibrate itself
    \item 
\end{itemize}

\subparagraph*{Basic requirements for arenas}

The following items have to fulfilled to be able to host an arena.

\begin{itemize}
    \item A field that is nearly of the size of a 3/4 field or larger.
    \item Field marks as stated in the rules section. 
    \item Two standard goals with nets.
    \item Wifi with standard SPL_A session, standard password and standard IP address and DHCP turned off.
    \item Remote network access via:
    \begin{itemize}
        \item VPN connection
        \item Mobile connection
        \item TeamViewer or remote desktop connection
    \end{itemize}
    \item Camera, tablet or laptop for on field online calibration with remote team
    \item Streaming setup to stream the game on YouTube/Discord/BigBlueButton/Zoom ..., what is available
    \item Latest Game Controller running.
    \item At least 4 working NAOs V6 per game.
    \item At least 4 balls.
    \item Assistants for remote setup and autonomous setup
    \item Referees
\end{itemize}

\subsection{Setup}
% Patrick

- Field dimensions \& lighting situation
    - All selected arenas have to provide a standardised json file of the field dimensions, and images of the field from different angles and day\&night times with focus on the lighting conditions.
    - This file has to be loaded from the software somehow. (Idea image is independant from field dimensions json)

- Arena access publishing
- Streaming setup  publishing

- Images storage
    - All autonmous teams and optional remote teams with own images have to store their image on a to be defined storage hoster (TUHH could provide a Nextcloud for this) at least 2 hours before the game starts.

- Remote Setup procedure
    % - https://writemd.rz.tuhh.de/Tu3JCxovRVunD_yZwBl5Vw#
    % - https://writemd.rz.tuhh.de/rQE55j51QzqO98ifHYF0cQ#
    % - https://writemd.rz.tuhh.de/2YL-Gkz7QX2_zeJyD5Y1jA#

- Wifi setup

\subsection{Rules}
% Arne
All rules from the current rule book apply, except for these changes:

- Starting position for robots
    - Robots start in their own half on the GC side of the field, height penalty spot

- Refereeing
    - Local team has to referee
    - Referee can prevent robot by crashing on ground if falling by catching it before
    - Team scheidet aus, wenn Laufen so schlecht, dass es zwei Spiele lange nur hingefallen ist

1. Only one field player is allowed per team. There is one replacement player.
2. The player is only allowed to stay in its own half except between the goal nets. Leaving this area results in a standard removal penalty (leaving the field).
3. The half time duration is 5 min from play (play-off-mode in GC). The game break is 5 minutes and no change of  the software is allowed. After the initial ready/set phases the game state remains in playing regardless of shot goals.
4. The game is performed with two balls on each side.
5. On each side the goal free kick positions are the starting points for the two balls.
6. Points will be counted by the referees and not by the GC.
7. Points can be scored by:
    - Shooting the ball in opponents half and it stops there (1 Point)
    - Shooting a goal (not own goal) (1 Point)
    - Touching the opponents player with the ball (0 Point)
8. If a player scores a goal (not own goal) the ball gets replaced by the referees on the starting point on its own side that is farthest from said player. The ball can directly be played again.
9. If a balls leaves the field the ball is placed on the corresponding kick in or goal free kick positions without applying the GC action (ball is instantaniously free).
10. If the score is equal for both teams the game duration gets extended for/by one minute. After each extension the score is evaluated again. At max 5 extensions are allowed. TODO: Final determination? Coin toss?
11. Teams playing with an autonomous player get a scoring factor of 2. Each point is multiplied by this factor.
12. Dive motions and wide stance is not allowed
13. Global Game Stuck

\subsection{Challenge execution}
% Arne
- Ladder system
    - KO system 
    - groups of 8 teams per Ladder
    - randomly assignement of teams to groups
    - losers of a match will play in additional ladders (To allow all teams to play at least 3 times)
    - Winners play games against each other
    - Timezones (Two games per day)
    - can only be finalized when the exact number of participants is known