\section{Remote Challenge: Autonomous Calibration}
\label{sec:AutonmousCalibration}

\subsection{Idea of Challenge}

Autonomous calibration is a challenge for the SPL to reduce the dependence on extensive calibration of robots before each individual SPL game.
This challenge tests a team's autonomous calibration software separately from the 1v1 challenges to allow teams to demonstrate the quality of their calibration tools.
Autonomous calibration may include adjusting:
\begin{itemize}
    \item Camera settings to dynamic lighting conditions and individual robots
    \item Locomotion parameters for stable movement on different fields surfaces and individual robots
%    \item Localisation to variance in field dimensions
%    \item Audio calibration to variance in whistles and noise around individual fields
\end{itemize}

Requirements to participate in this challenge are:
\begin{enumerate}
    \item Deploy your robot's software remotely or via a ready to install and use image
    \item Use of the chest/head button interface
    \item Use of the head button to denote different stages of the challenge
    \item Run the software without GameController or any Wifi connection
    \item Use Text-To-Speech for reporting the progress of the robot\footnote{The TC has discussed the possibility of Text-To-Speech. Softbank provides a \texttt{say} executable on the standard Nao image which can be used. The TC permits and will help teams to share how to enable speech-to-text like the V6 remote setup. However, if teams note persistent problems the reporting mechanism will be adjusted.}
    \item The host venue will also need a measuring tape
\end{enumerate}

\subsection{Prerequisites \& Setup}

The Prerequisites and Setup for this challenge will follow the same process as described in Sections~\ref{sec:c3_Prerequisites} and~\ref{sec:arean-org-setup} for the ``1 vs 1 Challenge``, with the following important changes:
\begin{enumerate}
    \item Autonomous Calibration must be performed, and will be triggered at the start of the challenge using the chest/head button interface. 
%    \item Teams may confirm with the referee that their software is running and the robot is able to stiffen prior to starting automatic calibration
    \item The network will be disabled once the software setup is complete
    \item During setup and up until the challenge commences, a \textit{camera shield} will be placed over the robot's camera. The camera shield may be a sheet of opaque paper, or an opaque bag.
\end{enumerate}

\subsubsection{Timing}
Ideally, the challenge will be conducted in parallel with the `1vs1 Challenge', as described in Section~\ref{sec:OneVsOneChallenge}.
This challenge will be conducted before teams \textit{manually} calibrate their robots for the 1vs1 challenge. 
%If this is not possible, the challenge will be conducted the first time that a team set's up their robots for use at a given venue.

\subsection{Challenge Execution}

The challenge is conducted in two phases:
\begin{enumerate}
    \item Autonomous Calibration
    \item Evaluation
\end{enumerate}

\subsubsection{Phase 1: Automatic Calibration}

The autonomous calibration phase will commence immediately after a team has confirmed the software has been successfully setup on the robot, and disconnected any from the robot.

The autonomous calibration phase will run as follows:
\begin{enumerate}
    \item The robot shall be placed at the normal initial position (that is, at the side-line, and in-line with the penalty spot). Two balls shall be placed on the field, one in each half, approximately in the centre of each half, clear of any fields lines.
    \begin{itemize}
        \item This shall denote the robot's  ``own half''.
        \item The other field half is the ``opposition half''.
        \item The camera shield shall remain in place while the robot is placed.
    \end{itemize} 
    \item The camera shield will be removed.
    \item The \textit{front} head button plus the chest button is pressed tapped. The head button press signifies to initialize auto calibration mode~(see~\ref{sec:robot_states}).
    \item The robot has up to 10 minutes to calibrate. The robot may freely move about the field. However it should not touch either ball, nor leave the boundaries of the field area.
    \item The robot may choose to end the calibration time early. It does so by standing still and raising one arm\footnote{If the arms are not functional, use of the arms can be forgone in agreement with the referee.}. It should also say "Calibration finished".
    \begin{itemize}
        \item If the robot finishes the calibration time early, the timer should be stopped and the total calibration time recorded.
        \item The time is rounded to the nearest 5 seconds to avoid human error
    \end{itemize} 
    \item If the 10 minute period is reached, the robot should immediately stop. The robot may also return to a seated position and unstiffen~(see~\ref{sec:robot_states}). If it does not do so the referee may use the chest button interface to penalise the robot (informing the robot to stop calibration). 
    \item The referee may optionally press all head buttons to unstiffen the robot, before processing to the evaluation phase.
    \item At the end of the automatic calibration phase, the camera shield is placed on the robot.
\end{enumerate} 

\subsubsection{Phase 2: Evaluation}

\begin{enumerate}
    \item The robot shall be placed in a stable position at the normal initial position
    \item Two balls shall be placed on the field, one in each half. The balls will be placed in a random position, such that
    \begin{itemize}
        \item The balls are within the field boundaries
        \item The balls are not on any field line
        \item The balls are fixed to the field with a small amount of tape to prevent them from moving.
        \item The positions will be provided to the hosting venue before the challenge commences
        \item All robot's participating at the same venue will be given the same fixed positions
    \end{itemize}
    \item The camera shield will be removed.
    \item If necessary, the head buttons will be pressed to unstiffen the robot~(see~\ref{sec:robot_states}).
    \item The \textit{back} head button plus the chest button is pressed. The head button press signifies the challenge mode, compared to a normal game/challenge mode.
    \item The referee will start a timer simultaneously with the above step.
    \item During the evaluation phase, the robot must complete 5 tasks in sequence as \textit{quickly} and \textit{accurately} as possible (see points 9/10 for details):
    \begin{enumerate}
        \item Report the \textit{field relative} (that is global) position of the ball in the robot's own half.
        \item Walk to the centre of the goal in it's own half, at any rotation
        \item Walk to the centre-circle ball kick-off spot, at any rotation
        \item Report the \textit{field relative} (that is global) position of the ball in the ``opposition'' half
        \item Walk to the centre of the goal in the ``opposition'' half, at any rotation
    \end{enumerate}
    \item The robot has up to 10 minutes to complete the challenge. The robot may freely move about the field. However it should not touch either ball, nor leave the field area.
    \item To report a ball position the robot should:
    \begin{enumerate}
        \item Raise one arm
        \item Say: ``The ball is at: x y'', where x and y is the \textit{global} co-ordinates of the ball position. The global co-ordinate reference frame is defined in the \texttt{SPLStandardMessage.h} header file\footnote{Linked \href{https://github.com/bhuman/GameController/blob/master/include/SPLStandardMessage.h}{here} for reference.}.
        \item The referee should record the stated position, rounded to the nearest 10cm.
    \end{enumerate}
    \item To report the desired position is reached the robot should:
    \begin{enumerate}
        \item Raise one arm
        \item Say: ``Finished walking''
        \item The referee should record the robot's position by making the robot's location with a small amount of tape
    \end{enumerate}
    \item When a robot has completed a task, referee should attempt to allow the robot to continue as quickly as possible. The referee does so by tapping \textit{any} head button. The referee should pause the timer between the robot completing a task, and the referee tapping the head buttons.
    \item When the robot completes the final task, the referee should stop the timer and record the total time taken in the evaluation phase, rounded to the nearest 5 seconds, to avoid human error 
    \item If the 10 minute period is reached or the robot has finished the final task, the chest button of the robot is pressed to penalise the robot and complete the evaluation phase.
    \item \textit{After} the conclusion of the evaluation phase, the referee should:
    \begin{enumerate}
        \item Measure the distance of the marked position of the robot to the correct position for each of the walking tasks, rounded to the nearest 10cm.
        \item Compute the distance from the reported ball position to the correct position for each of the ball tasks, rounded to the nearest 10cm.
    \end{enumerate}  
\end{enumerate} 

\subsubsection{Termination Conditions}

The challenge will be terminated at any stage if any of the following conditions are met:
\begin{itemize}
    \item The robot falls \textit{three} times
    \item The robot falls and is unable to return to a standing position after no more than 1 minute.
    \item The robot leaves the external boundary of the field or touches a ball.
    \item A robot takes a dangerous action that would result in damaging the robot. %This includes a robot obviously grinding gears in attempting to stand after failing.
\end{itemize}

\subsubsection{Scoring}

Scoring is conducted by ranking robots in each of the following categories:
\begin{enumerate}
    \item Fastest time to complete the automatic calibration phase
    \item Fastest time to complete the evaluation phase
    \item Closest reported position of the ball in the robot's ``own'' half.
    \item Closest reported position of the ball in the ``opposition'' half.
    \item Closest robot position at the goal in the robot's ``own'' half.
    \item Closest robot position at the centre circle
    \item Closest robot position at the goal in the ``opposition'' half.
\end{enumerate} 

\subsubsection{Venue Rankings}
Due to differences between venues, teams are ranked according to all teams that attempted the challenge at each venue. Teams are ranked separately in each of the 7 categories, with the top ranked team receiving a rank of 1, and other ranks increasing monotonically.

Teams receive a total score equal to the sum of their ranks in each category at the venue. 
Thus, the team with the \textit{lowest total score} is the best team at the venue.

\subsection{Combine Scoring and Final Winner}
Teams are given a final score equal to their \textit{average score} across all venues at which they attempted the challenge. A team must attempt the challenge at \textit{at least 3 venues} to qualify for the final ranking\footnote{The number of venues and completion of the challenge is subject to scheduling constraints and will be confirmed closer to the competition.}.
Places are awarded in the challenge teams in increasing average score, starting with the lowest scoring team receiving 1st place.

