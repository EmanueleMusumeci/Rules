\section{Introduction}

Owing to the ongoing COVID 19 situation restrictions on travel and group activities are in place in most countries around the world. Thus, RoboCup Standard Platform League Competition 2021 was designed to provide all teams an opportunity to measure their skills in robot soccer while taking into consideration the limitations that are placed onto all SPL teams. No travel to competition sites or other team's venues will be necessary for the 2021 competition and associated local competitions. Live streaming of videos will be incorporated into the event to keep the contact between teams alive as well as presenting an entertaining experience for the interested public and sponsors.

RoboCup SPL will host four challenges in 2021:

\begin{itemize}
    \item the first challenge is testing individual game play. 
    \item the second challenge is testing cooperative team coordination.
    \item the third challenge is testing remote competitive game play. 
    \item the fourth challenge is testing the quality of remote deployment and calibration.
\end{itemize}

The first two challenges allow teams to compete by exclusively using their own robots in their local venue. The latter two challenges ask teams to remotely deploy their code onto robots in other competition sites.

With RoboCup being a research competition, the past years have shown that SPL teams are great at finding a balance between competing with other teams to win awards and advancing the field of computing science by doing research and publications. Given the restrictions on any event in 2021 we ask all teams to do their best to make this year's RoboCup SPL succeed. One particular restriction that weighs on all teams are the limited opportunities to service and repair their robots. RoboCup 2021 was planned with the goal of avoiding robot breakage. This includes a call to all teams to be proactive in rules enforcement when acting as referees during the competition to prevent damage. As teams are asked to install their code onto robots of other teams specific care should be taken to not incur damage due to software bugs. Another aspect are the limitations on verifying the adherence to the rules as written. The TC is aware that hiding additional sensing equipment or remote control stations will be easier this competition. Given the long years that most of the teams are involved in the SPL and the close contact established between teams, the TC trust all teams to act fair during competition.

Conditions of the venues across all SPL teams tend to vary. Thus, in 2021 teams can choose to participate in some or all of the challenges of SPL. Details on registration procedures for all challenges will be published soon.

\subsection{Key dates}

\begin{itemize}[leftmargin=*,labelsep=0.7cm, labelindent=2cm]
    \item [2021-01-31] Publishing the first draft or rules with mandatory files.
    \item [2021-04-01] Commitment to participate
    \item [2021-05-01] Jersey approval for new or modified jerseys
    \item [2021-06-01] Submission of credentials for arena access to Teams
    \item [2021-06-08] Testing hours for arena access with remote teams
    \item [2021-06-15] Field\_dimension.json
    \item [2021-06-15] Arena network access responsible contact person
    \item [2021-06-22] Start of RoboCup 2021
\end{itemize}
