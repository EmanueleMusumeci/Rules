\section{Introduction}

In 2021, RoboCup will be held virtually around the world, owing to the ongoing COVID 19 situation. The Standard Platform League will take on a new form that is designed to provide all teams an opportunity to competitively measure their skills in robot soccer within the current circumstances. 
%No travel to competition sites or other teams venues will be necessary for the 2021 competition and associated local competitions. 
Live streaming of videos will be incorporated into the event to keep the contact between teams alive as well as presenting an entertaining experience for the public and sponsors.

For 2021, the RoboCup SPL competition will consist of four challenges of increasing complexity:

\begin{enumerate}
    \item Testing game play of an individual robot. 
    \item Testing team coordination.
    \item Testing remote competitive game play. 
    \item Testing the quality of remote deployment and calibration.
\end{enumerate}

Conditions of the home venues across all SPL teams tend to vary. Thus, these challenges allow teams to participate in some or all of the challenges depending on their capabilities. The first two challenges allow teams to compete at their home venue exclusively using their own robots. The latter two challenges ask teams to compete by remotely deploy their code onto robots at a neutral host venue, which is an important step towards facilitating fully remote soccer games.

The past years have shown that SPL teams are great at finding a balance between competing with other teams to win awards and maintaining the research focus on RoboCup in advancing the field of computing science through publications. Given the restrictions on any event in 2021 we ask all teams to do their best to make this year's SPL successful and enjoyable for everyone. 

\subsection{Code of Honour}

The TC trust all teams to act fair during competition. Thus we ask all participants to acknowledge a Code of Honour.
\begin{enumerate}
    \item The safe operation of the robots is paramount, especially where teams deploy remote their software onto robots owned by other teams. Referees and participants should act to prevent damage to robots, and be proactive in enforcing rules  designed to minimise damage.
    \item Teams should endeavour to install bug-free and conservative software for remote participation. While remote challenges features elements of speed, there are strict consequences for damaging robots, and we hope no team will be prevented from competing due to these reasons
    \item Teams and participants should be open and honest in the presentation of their local challenge entries and capability of their software.  A live stream should be used where possible, and pre-recorded videos uploaded as close as possible to the scheduled completion of the challenge. The use of features indicating date and time is encouraged.
    \item Teams (and host venues) should present a clear view of the field and surrounds. Teams should not use additional sensing equipment or remote control stations.
\end{enumerate} 

It is our experience that SPL teams are extremely gracious and congenial, and we expect that the Code of Honour will be honoured. However, violation of this code may result in immediate disqualification.

\subsection{Registration}

Details on registration procedures will be published and updated here, once a time frame is confirmed by the RoboCup Federation.

\subsection{Key dates}

\begin{itemize}[leftmargin=*,labelsep=0.7cm, labelindent=2cm]
    \item [2021-01-31] Publishing the first draft or rules with mandatory files.
    \item [2021-04-01] Commitment to participate
    \item [2021-05-01] Jersey approval for new or modified jerseys
    \item [2021-06-01] Submission of credentials for arena access to Teams
    \item [2021-06-08] Testing hours for arena access with remote teams
    \item [2021-06-15] Field\_dimension.json
    \item [2021-06-15] Arena network access responsible contact person
    \item [2021-06-22] Start of RoboCup 2021
\end{itemize}
